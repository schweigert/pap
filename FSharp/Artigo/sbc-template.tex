

\documentclass[12pt]{article}
\usepackage{listings}
\usepackage{sbc-template}

\usepackage{graphicx,url}
 
\usepackage[brazil]{babel}   
%\usepackage[latin1]{inputenc}  
\usepackage[utf8]{inputenc}  
% UTF-8 encoding is recommended by ShareLaTex

     
\sloppy

\title{Paradigmas de Programação \\ FSharp}

\author{Marlon Henry Schweigert, Jonathan Oliveira}


\address{Centro de Ciências Tecnológicas -- Universidade do Estado de Santa Catarina (UDESC)
}

\documentclass[a4paper,11pt]{article}

\usepackage{listings}
\usepackage{color}

\definecolor{bluekeywords}{rgb}{0.13,0.13,1}
\definecolor{greencomments}{rgb}{0,0.5,0}
\definecolor{turqusnumbers}{rgb}{0.17,0.57,0.69}
\definecolor{redstrings}{rgb}{0.5,0,0}

\lstdefinelanguage{FSharp}
                {morekeywords={let, new, match, with, rec, open, module, namespace, type, of, member, and, for, in, do, begin, end, fun, function, try, mutable, if, then, else},
    keywordstyle=\color{bluekeywords},
    sensitive=false,
    morecomment=[l][\color{greencomments}]{///},
    morecomment=[l][\color{greencomments}]{//},
    morecomment=[s][\color{greencomments}]{{(*}{*)}},
    morestring=[b]",
    stringstyle=\color{redstrings}
    }

\begin{document} 
\lstset{language=FSharp}

\maketitle

     
\begin{resumo} 
Este documento apresenta as principais características da linguagem de programação FSharp, sendo base para o seminário final da matéria de Paradigmas de Programação, ministrado pelo Professor Doutor Cristiano Damiani Vasconcellos.
\end{resumo}


\section{Introdução}

A linguagem de programação FSharp foi desenvolvida por Don Syme, nos laboratórios da Microsoft Research, com o objetivo de atender os 3 principais paradigmas de programação utilizando as bibliotecas .NET: Programação Funcional, Programação Imperativa e Programação Orientada a Objetos.


\section{FSharp}

Tendo sua performance igualada a C++ e OCaml, pertencendo a família das linguagens ML (ao lado de CSharp, Visual Basic, entre outros), conta com segurança e inferência de tipos, sendo uma forma eficiênte de programação funcional com as bibliotecas do .NET Framework. Como citado por Yana Kellen \cite{yanakellen}, o seu desenvolvimento atual é, grande parte, com foco em programação paralela, sendo fortemente utilizada na programação financeira quantitativa (antes dominado pelo Cobol), comercio de energia e jogos para Facebook.

A linguagem pode ser compilada ou interpretada. Ambos, compilador e interpretador, podem ser baixados no site oficial da linguagem \cite{fsharp}. Para distros ubuntu e debian, o PPA do compilador já está incluso. Um alternativa de compilador é o projeto MONO. Um projeto de compilador que surgiu com a ideia de bater de frente com o pacote de compiladores da Microsoft.

Atualmente o compilador da linguagem é OpenSorce e mantida pela Microsoft e MONO, em uma parceria realizada no inicio de 2016.

\section{Critérios}

Os principais critérios do FSharp em sua Sintaxe são:

- Legibilidade: Próximo de uma linguagem matemática;

- Redigibilidade: Indentação mais econômica, economia de linhas desnecessárias como blocos, estruturas. Usa a tabulação como auxilio para a sua economia, como Python e Haskell.

- Confiabilidade: Programas podem ser facilmente verificados;


\section{Características}

As principais características em FSharp é a sua simplicidade, ortogonalidade, portabilidade, expressividade, e reusabilidade. Cada aspecto citado é levado em conta em sua sintaxe e maneira de utilizar.

- Simplicidade: Maior simplicidade Sintática.

- Ortogonalidade: Um nome pode tanto receber um valor final, quanto uma função de avaliação "curried".

- Portabilidade: Beneficía-se do .NET Framework / MONO;

- Expressividade: Vantagens da OO somada a Natureza declarativa da programação funcional;

- Reusabilidade: Conta com vantagens funcionais;

\subsection{Exemplos de Código-Fonte}

Exemplo 1: Olá mundo
\begin{lstlisting}
    printfn "Ola mundo"
\end{lstlisting}
Exemplo 2: Declarações
\begin{lstlisting}
    let a = 1
    let b = 3u
    let c = "texto pequeno"
\end{lstlisting}    
Exemplo 3: Tuplas e Listas
\begin{lstlisting}
    let tupla = ("direita", "esquerda")
    let lista = [1;2;3;4;5]
 \end{lstlisting}     
Exemplo 4: Recursividade
\begin{lstlisting}
    let rec fat n =
        if n = 0
        then 1
        else n * fat (n-1)
 \end{lstlisting}  
Exemplo 5: Fibonacci
 \begin{lstlisting}
(* Pela Fórmula *)
let rec fib n =
    match n with
    | 0 | 1 -> n
    | _ -> fib (n - 1) + fib (n - 2)
 
(* Utilizando zip *)
let rec fibs = seq {
    yield! [1; 1];
    for (x, y) in Seq.zip fibs (Seq.skip 1 fibs) -> x + y }
 
(* Utilizando |> *)
[1 .. 10]
|> List.map     fib
|> List.filter  (fun n -> (n % 2) = 0)
|> printlist
 
(* O mesmo, usando repetição *)
[ for i in 1..10 do
    let r = fib i
    if r % 2 = 0 then yield r ]
|> printlist 
 
 \end{lstlisting}
Exemplo 6: Funções Lambda
\begin{lstlisting}
let add = (fun x y -> x + y)
add 2 2


let twoTest test =
    test 2
twoTest (fun x -> x < 0)


let firstHundred = [0..100]
let doubled = List.map (fun x -> x * 2)

\end{lstlisting}
Exemplo 7: Listas

\begin{lstlisting}
List.sum (List.map (fun x -> x * 2) 
    (List.filter (fun x -> x % 2 = 0) [0..100]))
\end{lstlisting}
Exemplo 8: One line Source

\begin{lstlisting}

// Composicao de funcao

[0..100] // Pega uma lista de 0 - 100
|> List.filter (fun x -> x % 2 = 0)  // Aplica um filtro
|> List.map (fun x -> x * 2) // Multiplica por 2
|> List.sum // Retorna o somatorio do resultado
\end{lstlisting}

Este e muitos outros exemplos podem ser encontrados no site Try F# \cite{tryfs}.




\bibliographystyle{sbc}
\bibliography{sbc-template}

\end{document}
\end{lstlisting}
